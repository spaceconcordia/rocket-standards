\subsection{Why you need more checklists}
    \begin{itemize}
        \item Once you hit the desert, assume that everyone loses the ability to think. It's 40 degrees, everyone has heatstroke, and they're probably operating at about the same level as your standard alcoholic. It's much harder to make dangerous mistakes when following a checklist.
    \end{itemize}

\subsection{How to do a great checklist}
    \begin{itemize}
        \item Know your failure modes, know at what point they could arise during your checklist, and write down the appropriate abort procedure to take to minimize risk at that failure mode.
        \item No step in your procedure is optional. If it’s written down, it’s mandatory. If it’s not mandatory, it’s unnecessary and should be removed.
        \item Have procedures for every off-nominal scenario, regardless of how unlikely it is. You should do this for two reasons: 
        \begin{enumerate}
            \item There is always a possibility that you’ll end up in that scenario anyways, and you want to be prepared rather than scrambling to figure out what you should do next.
            \item Creating and reviewing off-nominal procedures forces you to become more familiar with the procedure as a whole, and lets you identify anything that you may have overlooked or any mistakes you might have made.
        \end{enumerate}
        \item Once you have a complete checklist, read it out with all people who are going to be involved in it. If you notice anything wrong or potentially unsafe in your checklist, change it. Then read it out with all your people again. Spending an extra few hours working at home is worth it if it makes desert operations run smoother and safer.
        \item If your primary plan puts human being inside the Minimum Safe Distance (2000 feet) while your rocket is in an unsafe state (solid engine: ignition leads are hooked up, hybrid engine: Ignition leads are hooked up or your rocket has any amount of pressurized fluid in it), then you have fucked up. Rewrite it. If you have to make changes to other parts of your system, then do so. Rockets are dangerous, and the best way of minimizing their risk is staying away from them while they are in a dangerous state.
    \end{itemize}

\subsection{Launch Operations}
    This section mostly applies to hybrid rockets, which require additional steps due to filling procedures.
    \begin{itemize}
        \item Have defined roles for everyone who will be part of operations procedures. Before you even get to the desert, you need to know exactly how many launch technicians will be necessary, who those technicians are going to be, and what each technician is going to do.
        \item Your operations procedure must unambiguously describe every action that will be performed by every technician, and identify who (by role) will be performing it. You don’t want to start filling your rocket and then realize that you don’t have enough people to finish the process.
        \item Every technician involved in launch operations should be familiar with every step that they need to perform - both nominal and off-nominal. Do at least one read-through of the procedure with everyone involved, and at least one dry run or simulated launch that gets as close to actual procedures as possible without placing the rocket in an unsafe state. Make any changes that are necessary, and practice any parts that get changed until everything runs perfectly. You want to eliminate any potential issues now, rather than trying to improvise while your rocket is filling with oxidizer.
        \item Exactly one person should be given control of operations during launch procedures, and identified as such (Flight Director, Technician In Control, etc.). This person’s only role should be to read out each step of the operations procedure to the technicians responsible for performing it, and to proceed to an off-nominal or abort procedure if necessary. Your flight director should be familiar with every single step of the procedure, and should have a physical copy in front of them during fill and launch operations.
        \begin{itemize}
            \item I can’t stress enough that the flight director should only be responsible for coordinating the procedure and directing the technicians. Hybrid launch operations are complicated and time-sensitive, and if you accidentally miss a single step, you could easily both jeopardize your chances of a nominal flight and put lives in danger. The flight director needs to be able to concentrate fully on the procedure. That means that you should have a separate technician (or technicians) operating any ground stations, including ignition.
        \end{itemize}
        \item If, at any point during your launch procedure, you have personnel in multiple locations, the flight director must have radio communication with them. If communication is lost, you need to stop operations immediately. You don’t want to be sending an ignition signal when your technicians are at the pad standing 20 feet from the rocket.
        \item Perform all relevant checks before you start launch operations, rather than during. Have a checklist of checks to perform before starting the procedure. This should include things like:
        \begin{itemize}
            \item Ensure your technicians have radios 
            \item Electrical checks on ground stations and launch systems
            \item Make sure the range is clear
        \end{itemize}
        \item During launch operations, the flight director should give instructions by first identifying the technician who will perform the action and then by stating the instruction (turn this valve, make this electrical connection, press this button, etc.). The flight director should wait for confirmation that the action has been successfully performed before continuing. Launch ops are stressful and can be hectic. You don’t want things to go wrong because one of your technicians misheard you or couldn’t keep up with the pace you were reading.
    \end{itemize}

\subsection{How to survive in the desert}
    Each person should have all the items listed below. The items should be checked by a team safety officer or supervisor prior to going in-field
    \begin{enumerate}
    \item A wide-brim hat or baseball cap with a covering for the back of the neck
    \item Sunscreen, 70 spf or higher (each individual must have a bottle)
    \item Water, at least 5 liters each (or a sports drink like Powerade)
    \item Team shirt
    \item Long pants (light cotton suggested)
    \item Closed toe shoes with socks
    \item Safety glasses (preferably tinted)
    \item Bandanas or something to cover the face
    \end{enumerate}

    The following equipment should be brought to the desert and only one item is necessary for the entire team. The responsibility for making sure this equipment should fall on the team captain or relevant safety officer
    \begin{enumerate}
        \item Properly sized tent
        \item A number of chairs equivalent to the number of team members in the desert
        \item At least 2 tables
        \item Additional water, at least 3 cases of 24 water bottles. A sports drink like Powerade can also be used
        \item Additional sunscreen
        \item Food
        \item Cars with sufficient space to carry all team members to a from the launch site. These vehicles should have air conditioning, they should not be convertibles and they should not be too low to the ground
        \item A large first aid kit
        \item Face Shields
        \item Salt
    \end{enumerate}

    The following outlines some of the greatest hazards presented by the desert environment. Remember, never, ever be alone in the desert. Make sure you are always with someone.
    \begin{enumerate}
        \item Blowing dust: High winds are often present in the desert. These high winds can create a significant amount of blowing dust or sand. If these conditions are encountered, the safety goggles and bandanas should be used to protect the face
        \item Strong Sunlight: Always use sunscreen and a wide-brim hat
        \item Extreme heat and dryness: The temperature can rise up to 45 degrees Celsius with virtually no humidity. In these conditions, heat stroke and dehydration are real and present dangers. Light clothing must be worn, so as to avoid overheating. As well, copious amounts of water must be consumed. If you are not urinating, then you are not drinking enough water. Finally, constant perspiration can deplete the amount of salt present in the body. Thus, a large amount of salt must be consumed or sports drinks must be consumed along with water. If you begin to feel dizzy or lightheaded, sit down in the shade and drink large quantities of water.
        \item Poisonous Animals: The desert is home to animals such as scorpions, tarantulas, and rattlesnakes. Avoid these at all costs. Take extra care when climbing through jumbled rocks. Avoid turning over rocks unless it is with a long stick. If bitten by one of these creatures, report to paramedics or emergency services immediately. If you cannot positively identify the creature, take a picture of it so it can be more easily recognized by medical personnel who can then administer the appropriate antivenin. If Killer Bees are encountered, run away as fast as possible.
        \item Cacti: The desert has many forms of cacti, all of which have spines and needles which could cause significant discomfort. Wear long pants and high boots to avoid having needles impaled into your legs. If you do end up with cactus spines embedded in you skin, remove with pliers and treat the wound like any other cut.
        \item Additional Competition Safety to be observed at the launch site:
        \begin{enumerate}
            \item Driving: Do not drive or stay in a car while there is a rocket about to be launched or in the air. Range safety officers will keep you informed as to the current safety condition of the camp.
            \item Rocket Emergencies: If you hear the call “Heads-up!” or hear an air raid siren, then this means that there is a critical safety hazard which has been caused by a flying rocket. Either it has gone ballistic or has broken apart causing a significant amount of debris. Immediately exit from any form of shelter and watch the sky. Listen for any calls from range safety officers.
            \item Work Areas: Do not enter any work areas without proper safety equipment and authorization. Do not enter a work area that does not belong to your team unless invited to do so.
            \item Desert Recovery: When venturing into the desert to recover a rocket, remember to bring adequate water and gear. Do not venture into the desert without a certified BARQ radio operator. The radio operator is necessary as he carries a GPS as has constant contact with basecamp. Thus, he can easily guide the team back to base and stay aware of any hazards announced by Range Safety Officers. Make sure all personnel venturing into the desert are of adequate physical fitness to make the trip.
            \item Additional Safety Hazards: If you see another team failing to follow proper safety guidelines (such as improper protection when working with black powder) inform them of their error. If they do not correct it, inform a Range Safety Officer.
        \end{enumerate}
    \end{enumerate}
