\subsection{General Safety Guidelines}
\begin{itemize}
\item This system stops your rocket from falling on human beings at terminal velocity. If there is anything more than an infinitesimal chance that it will fail, then you are recklessly endangering human lives. This system is by far the most safety critical.
\item Test it. Make sure it doesn't just work, make sure it works flawlessly. If it doesn't work flawlessly under laboratory conditions, It’s possible that it will function even worse in the field.
\item This goes without saying for all points in this document, but bears explicit statement here: IREC publishes their own set of standards for operation and testing of many flight components. While these (the Canadian Rocketry) guidelines are not binding, theirs (IREC’s) are, and thus must be met or exceeded (go for exceeded. Many of their standards are bare minimum, and you can probably do better)
\end{itemize}

\subsection{Ground Testing of Recovery Systems}
The way to properly validate the system by which a parachute is ejected from a rocket is the recovery ground test. The following is a generalized guideline for performing a recovery ground tests safely and assessing the results to determine if your system is functional.
\par
The minimum safety equipment requirements for any personnel directly involved in testing are listed below. Personnel directly involved in the test are defined as personnel who are responsible for handling the packing of any pyrotechnic charges, or being within 30 feet of the rocket when it is loaded with any pyrotechnic substances or has any stored-energy.
\begin{itemize}
\item Safety Glasses
\item Face Shields (if it is expected that the test may produce debris)
\item Flame-resistant gloves (if parts are expected to be hot after the test)
\item Closed Toe Shoes
\item Long pants and Long Sleeved Shirt (if it is expected that the test may produce debris)
\end{itemize}

The following requirements for the test site should be observed:
\begin{itemize}
\item The rocket must be placed on a stand of some kind and should be kept horizontal. A hanging cradle for the rocket is also acceptable, but both halves of the rocket must be attached to the cradle and must not become detached from the cradle during the test.
\item It is advisable that any solid part being ejected be covered in EVA foam or the ground where it will land be covered in foam (for the protection of the part)
\item The area 30 feet in front and behind the rocket should be cleared of people.
\item Non-essential people should not be in the area where parts will be ejected and should preferably be 15 feet away.
\item The area should be cleared of any flammable material if pyrotechnics are being used for ejection
\item There must be no ignition sources in the area if pyrotechnics are being used for ejection (that means no smoking)
\item A fire-extinguisher must be kept on hand if pyrotechnics are being used for ejection
\end{itemize}
